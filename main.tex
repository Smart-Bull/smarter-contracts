\documentclass{article}
\usepackage[utf8]{inputenc}
\usepackage{amsmath,amssymb}
\usepackage{makecell}
\usepackage[a4paper, total={6in, 9in}]{geometry}

\title{Smart Contract Signals and Slots Formalization}
\date{May 2020}

\begin{document}

\maketitle


\section{Introduction}
This document presents the formal definitions behind signals and slots in a blockchain execution environment. A familiarity with the Ethereum Yellow Paper is expected prior to reading this document. In this section, we highlight a few important values and functions.
\begin{itemize}
    \item $\sigma$: world state
    \item $\mu$: machine state
    \item $\alpha$: account address
    \item \texttt{KEC()}: Keccak-256 hash function
    \item \texttt{RLP()}: Recursive length prefix serialization
    \item \texttt{TRIE()}: Returns the root value of the trie 
\end{itemize}


\section{Signals and Slots}
We denote signals as $E$ with the following fields:
\begin{itemize}
    \item owner, $E_a$: The address of the contract that this signal belongs to.
    \item identifier, $E_{i}$: A unique identifier associated with a signal generated during contract creation. By assigning each signal a unique ephemeral $sigLocalId$ during contract creation, \texttt{KEC}($E_a + sigLocalId$) can be used to generate a unique identifier.
    \item data, $E_d$: An arbitrary size byte array containing the output data of the signal.
\end{itemize}
We denote slots as $L$ with the following fields:
\begin{itemize}
    \item owner, $L_a$: The address of the contract that this slot belongs to. This is also the address that pays for the gas consumed during the execution of the slot.
    \item code, $L_c$: A pointer to executable EVM code that is the entry point to the slot.
    \item gasLimit, $L_g$: A scalar value equal to the maximum amount of gas that should be used in executing this slot.
    \item nonce, $L_n$: A scalar counter recording the number of previous call to the slot. 
\end{itemize}
Note: each $L$ can be attached to only one $E$ while each $E$ can be listened by multiple $Ls$.  
    
\section{World State}
All entities have their necessary information stored in an account, represented by a 20-byte address $\alpha$. The world state is a mapping between addresses and account states. An account state $\sigma[\alpha]$ has the following five fields:
\begin{itemize}
    \item nonce, $\sigma[\alpha]_{n}$: A scalar counter recording the number of previous activities initialized by this account.
    \item balance, $\sigma[\alpha]_{b}$: A scalar value representing the number of Wei owned by this account. 
    \item storageRoot, $\sigma[\alpha]_{s}$: Hash of the root node of the trie that encodes the storage content of this account.
    \item codeHash, $\sigma[\alpha]_{c}$: Hash of the EVM code that gets executed when $\sigma[\alpha]$ receives a message call. This is immutable once established. 
    \item slotRoot, $\sigma[\alpha]_{l}$: Hash of the root node of the trie that maps $E_{i}$ to \texttt{RLP}(
    \texttt{LIST}($L$)). 
    \item slotCount, $\sigma[\alpha]_{lc}$: A scalar counter recording the number of queued slot transactions pointing to this address. Normal transactions are blocked unless this counter is zero.
\end{itemize}
Therefore an account state $\sigma[\alpha]$ can be represented as the following tuple:
\begin{equation*}
    \sigma[\alpha] \equiv (\sigma[\alpha]_n, \sigma[\alpha]_b, \sigma[\alpha]_s, \sigma[\alpha]_c, \sigma[\alpha]_l, \sigma[\alpha]_{lc})
\end{equation*}


\section{Slot Transaction}
Slot Transaction $ST$ is a single instruction resulted from a signal instance at an emitter contract $\sigma[\alpha_{emitter}]$ that are directed to some slot in a listener contract $\sigma[\alpha_{listener}]$. These transactions do not require signature from either party because its validation is done according to the signal trie maintained in the storage. The fields included in a special transaction are:
\begin{itemize}
    \item nonce, $ST_n$: A scalar value equal to the number of previous ST from the signal and handler combination. 
    \item signal, $ST_e$: A pointer to the signal tuple $E$ of interest.
    \item slot, $ST_l$: A pointer to the slot tuple $L$ of interest.
\end{itemize}


\newpage
\section{Block Header}
Currently, every Conflux block $B$ consists of two parts: a block header $H$ and a list of transactions $Ts$. On top of this block structure, we are adding list of special transactions, $STs$. The block header $H$ is a collection of relevant pieces of information:
\begin{itemize}
    \item parentHash, $H_p$: Keccak 256-bit hash of the parent block’s header.
    \item refereeHash, $H_o$: serialized RLP sequence of the referee list consisting of Keccak 256-bit hashes of referee blocks.
    \item author, $H_a$: address of the author.
    \item transactionRoot, $H_t$: Keccak 256-bit hash of the root node of transaction trie.
    \item deferredStateRoot, $H_r$: Keccak 256-bit hash of the root node of the state trie after “stable transactions” are executed.
    \item deferredReceiptsRoot, $H_e$: Keccak 256-bit hash of the root node of the receipt trie during the construction of deferredStateRoot.
    \item deferredLogsBloom, $H_b$: bloom filter for logs of transactions receipts included.
    \item blame, $H_m$: A scalar value for evaluating ancestor blocks.
    \item difficulty, $H_d$: Value corresponding to the difficulty of the block.
    \item number, $H_i$: A scalar value equal to the number of ancestor blocks.
    \item adaptiveWeight, $H_w$:
    \item height, $H_h$: number of parent references to reach the genesis block.
    \item gasLimit, $H_l$: scalar value to the current limit of gas expenditure per block.
    \item timestamps, $H_s$: Unix time.
    \item nonce, $H_n$: Value that proves that a sufficient amount of work has been carried out on this block.
    \item slotTransactionRoot, $H_{st}$: Hash of the root node of slot handler trie, $H_{st}$. The trie is a mapping of \texttt{KEC}($slotMinHeight$) to \texttt{RLP}(\texttt{FIFO}($ST$))). $slotMinHeight$ indicatess the minimum height the corresponding $STs$ can be called. Once all $STs$ at a certain $slotMinHeight$ are proceeded, the corresponding leaf can be pruned from the trie.
\end{itemize}
Therefore the block $B$ can be represented as follows:
\begin{equation*}
    B \equiv (B_H, B_{Ts}, B_{STs})
\end{equation*}


\newpage
\section{Execution Environment}
The list of opcodes we need for implementing the proposed event-driven smart contract design. Borrowing the notation from the Ethereum Yellow Paper, we assume $O$ is the EVM state-progression function and define the terms pertaining to the next cycle’s state ($\sigma$, $\mu$) such that:
\begin{equation*}
    O(\sigma, \mu, A, I) \equiv (\sigma', \mu', A', I)
\end{equation*}
where $\sigma$ represents the active memory or the system state, $\mu$ is the storage used, $A$ is the accrued substate (information acted upon immediately following the transaction), and $I$ is some pieces information used in the execution environment. 

The list of information is as listed below\\

\begin{tabular}{l|p{10cm}}
\textbf{Variable} & \textbf{Description} \\
\hline
\hline
$A_s$ & the self-destruct set: a set of accounts that will be discarded following the transaction’s completion \\
$A_l$ & log series \\
$A_t$ & touched accounts \\
$A_r$ & the refund balance \\
$\mu_\mathbf{s}$ & machine’s stack \\
$\mu_\mathbf{m}$ & machine’s memory \\
$\mu_i$ & the active number of words in memory (counting continuously from position 0)\\
$\mu_g$ & gas available\\
$\mu_{pc}$ & the program counter\\
$I_a$ & the address of the account which owns the code that is executing \\
$I_o$ & the sender address of the transaction that originated this execution \\
$I_p$ & the price of gas in the transaction that originated this execution \\
$I_d$ & the byte array that is the input data to this execution; if the execution agent is a transaction, this would be the transaction data \\
$I_s$ & the address of the account which caused the code to be executing \\
$I_v$ & the value, in Wei, passed to this account as part of the same procedure as execution \\
$I_b$ & the byte array that is the machine code to be executed. \\
$I_H$ & the block header of the present block \\
$I_e$ & the depth of the present message-call or contract-creation (i.e. the number of CALLs or CREATEs being executed at present)\\
$I_w$ & the permission to make modifications to the state\\
$I_{si}$ & a signal emitted \\
$I_h$ & a handler that need to be attached \\
\end{tabular}


\newpage
\section{Slot Transaction Execution}
This section formalizes how slot transactions are executed. Firstly a slot transaction is popped off $H_{st}$ and the addresses listener changes state. Next, a gas price and limit are determined and an upfront cost is charged to the slot account. Finally, an execution environment is set up and executed in the same way as a regular transaction.
\\\\
\textbf{State Change}: The following state changes occur to execute the slot. 
\\\\
$GET\_ST:$\\
$curHeight = min(H_{h}, H_{st}.keySet)$\\
$ST = H_{st}[curHeight].\texttt{DEQUEUE}()$\\
%$if(ST = \varnothing \land H_{sh} \leq H_i) \{ H_{sh}+=1; JUMP\ GET\_ST \}$\\\\
$if(ST \neq \varnothing)\{\\$
\phantom{x}\hspace{3ex} $\sigma'[\{ST_l\}_a]_{sc} = \sigma[\{ST_l\}_a]_{sc} - 1$\\
\phantom{x}\hspace{3ex} $ST_n = \{ST_l\}_n$\\
\phantom{x}\hspace{3ex} $\sigma'[\{ST_l\}_n] = \sigma[\{ST_l\}_n] + 1$\\
\phantom{x}\hspace{3ex} $if(H_{st}[curHeight].empty)\{\sigma'[H_{st}[curHeight]] = \varnothing\}$\\
$\}$\\
return $ST$
\\\\
\textbf{Gas Price}: To execute a slot transaction, we need to determine the gas price $I_p$. We calculate this by multiplying the average gas price of regular transactions in the previous block by $SlotTransactionGasRatio$. Let the previous block be denoted as $B'$, hence the transactions in the previous block is $B_{Ts}'$.
\begin{equation*}
I_p = SlotTransactionGasRatio \cdot \frac{\sum_{T \in B_{Ts}'} T_p}{|B_{Ts}'|} 
\end{equation*}
\\
\textbf{Gas Limit}: The gas limit is set to $\{ST_l\}_g$.
\\\\
\textbf{Intrinsic Gas}: Intrinsic gas $g_0$ is calculated as follows: 
\begin{equation*}
g_0 = 
\begin{cases}
G_{txdatazero}    &\text{if } \{ST_e\}_d=\varnothing,\\
G_{txdatanonzero} &\text{otherwise.}
\end{cases}
\end{equation*}
\\
\textbf{Up-front Cost}: Upfront cost $v_0$ is calculated as:
\begin{equation*}
    v_0 \equiv \{ST_l\}_g * I_p
\end{equation*}
\\
\textbf{Remaining Gas}: Remaining gas $g$ for computation is:
\begin{equation*}
    g = \{ST_l\}_g - v_0
\end{equation*}
\\
\textbf{Slot Transaction Validity}: The validity of an ST can be checked in much a similar way to regular transactions.
\begin{align*}
    ST_l \ &\neq \ \varnothing \ \ \land \\
    \sigma[\{ST_l\}_a] \ &\neq \ \varnothing \ \ \land \\
    ST_n \ &> \ ST_n' \ \ \forall \ \ \{ST': ST'\in B_{STs} \land ST_l' =  ST_l\} \ \ \land \\
    g_0 \ &\leq \ \{ST_l\}_g \ \ \land \\
    v_0 \ &\leq \ \sigma[\{ST_l\}_a]_b \ \ \land \\
    \{ST_l\}_g \ &\leq \ B_{H1} - l(B_R)_u
\end{align*}
\\
\textbf{Regular Transaction Validity}: The validity check of a regular transaction is changed slightly to accommodate slots. Note that in the Ethereum Yellow Paper, the address of transaction $T$ is denoted as $S(T)$. Because $S$ is used a lot in this document, the address of transaction $T$ is referred to as $T_a$.
\begin{align*}
    T_a \ &\neq \ \varnothing \ \ \land \\
    \sigma[T_a] \ &\neq \ \varnothing \ \ \land \\
    T_n \ &= \ \sigma[T_a]_n \ \ \land \\
    g_0 \ &\leq \ T_g \ \ \land \\
    v_0 \ &\leq \ \sigma[T_a]_b \ \ \land \\
    T_g \ &\leq \ B_{H1} - l(B_R)_u \ \ \land \\
    \sigma[T_a]_{lc} \ &= \ 0 \\ 
\end{align*}
\\
\textbf{Execution Environment}: With the above information, an execution environment can be initialized. Once the execution environment is set up, it can be executed like a normal transaction.
\begin{itemize}
    \item $I_a$, set to $\{ST_l\}_a$.
    \item $I_o$, set to $\{ST_e\}_a$.
    \item $I_p$, calculated above.
    \item $I_d$, set to $\{ST_e\}_d$.
    \item $I_s$, set to $\{ST_e\}_a$.
    \item $I_v$, set to $0$.
    \item $I_b$, set to $\sigma[\{ST_l\}_a]_c$.
    \item $I_h$, the block header of the present block.
    \item $I_c$, set to $\varnothing$.
    \item $I_w$, given permission to change state.
\end{itemize}
The machine state is set up as follows:
\begin{itemize}
    \item $\mu_s$, set to $\varnothing$.
    \item $\mu_m$, set to $\varnothing$.
    \item $\mu_i$, set to $0$.
    \item $\mu_g$, calculated above to be $g$.
    \item $\mu_{pc}$, set to $\{ST_l\}_c$.
\end{itemize}


\newpage
\section{BINDSIG and EMITSIG Opcodes}
Note: $\delta$ is the number of items required on the stack for a given operation, $\alpha$ is the number of items returned/added on the stack for a given operation. 

\begin{tabular}{l|l|l|p{9cm}}
\textbf{Opcode} & $\delta$ & $\alpha$ & \textbf{Description} \\
\hline
\hline
\makecell{BINDSIG} & \makecell{5} & \makecell{1} & \makecell[l]{
This opcode binds a listener to a signal specified with its sigId.\\ It binds a new leaf to the slot trie of the emitter contract. \\
\begin{tabular}{l|l}
\textbf{item on stack} & \textbf{Description} \\
\hline
0 & emitter contract address \\
1 & sigId \\
2 & slotPtr \\
3 & gasLimit \\
\end{tabular}\\
if $\sigma[I_a]\geq storageDeposit$: \\
\phantom{x}\hspace{3ex} $\sigma' = \sigma$, except\\
\phantom{x}\hspace{6ex} $\sigma'[\mu_s[0]]_l = \sigma[\mu_s[0]]_l[\mu_s[1]].\texttt{INSERT}((I_a, \mu_s[2], \mu_s[3]))$\\
\phantom{x}\hspace{6ex} and $\mu_s'[0]=1$\\
else:\\
\phantom{x}\hspace{3ex} $\sigma' = \sigma$,
$\mu_s'[0]=0$
}\\
\hline
\makecell{EMITSIG} & \makecell{5} & \makecell{1} & \makecell[l]{
This opcode creates an instance for the signal specified with its\\ sigId. It binds a new leaf to the signal trie. \\
\begin{tabular}{l|l}
\textbf{item on stack} & \textbf{Description} \\
\hline
0 & sigId \\
1 & number of block delayed \\
2 & pointer to signal data byte array \\
3 & number of elements in signal data byte array  \\
\end{tabular}\\
if $\sigma[I_a]\geq storageDeposit$: \\
\phantom{x}\hspace{3ex} $\sigma' = \sigma$, except\\
\phantom{x}\hspace{6ex} $\forall L \in \sigma[I_a]_l[\mu_s[0]]:$\\
\phantom{x}\hspace{9ex} $H_{st}[I_H_i + \mu_s[1]].\texttt{ENQUEUE}(\texttt{RLP}((I_a, \mu_s[0], (I_b[\mu_s[2] + 0], \dots I_b[\mu_s[2] + \mu_s[3]])), L)$\\
\phantom{x}\hspace{9ex}  $\sigma'[L_a]_{lc} = \sigma[L_a]_{lc}+ 1$ \\
\phantom{x}\hspace{6ex} and $\mu_s'[0]=1$\\
else:\\
\phantom{x}\hspace{3ex} $\sigma' = \sigma$,
$\mu_s'[0]=0$
}
\end{tabular}

\end{document}
